% Formato para un capítulo cualquiera

%Título del capítulo
\chapter{Códigos de estado HTTP}\label{appHttp}
Esta sección mostrará la lista de códigos de respuesta HTTP y frases estándar asociadas, destinadas a dar una descripción corta del estatus de la operación HTTP que se ha realizado. Estos códigos de estatus están especificados por el RFC 2616\footnote[1]{Hypertext Transfer Protocol -- HTTP/1.1} y algunos fragmentos en los estándares RFC 2518\footnote[2]{HTTP Extensions for Distributed Authoring -- WEBDAV}, RFC 2817\footnote[3]{Upgrading to TLS Within HTTP/1.1}, RFC 2295\footnote[4]{Transparent Content Negotiation in HTTP}, RFC 2774\footnote[5]{An HTTP Extension Framework} y RFC 4918\footnote[6]{HTTP Extensions for Web Distributed Authoring and Versioning (WebDAV)}; otros no están estandarizados, pero son comúnmente utilizados.
\bigskip
\par
El primer dígito de un código de estado HTTP especifica una de las siguientes clases de respuestas:
\begin{enumerate}
\item Respuestas informativas(1xx)
\item Peticiones correctas(2xx)
\item Redirecciones(3xx)
\item Errores de cliente(4xx)
\item Errores de servidor(5xx)
\end{enumerate}
\bigskip
\par
En éste apéndice se enumerarán y se explicarán todas las anteriores.

\section{Respuestas informativas(1xx)}
Este tipo de respuestas indicarán el siguiente mensaje: \textbf{Petición recibida, continuando proceso}.
\bigskip
\par
Esta clase de código de estatus indica una respuesta provisional, que consiste únicamente en la línea de estatus y en encabezados opcionales, y es terminada por una línea vacía. Ya que HTTP/1.0 no definía códigos de estatus 1xx, los servidores no deben enviar una respuesta 1xx a un cliente HTTP/1.0, excepto en condiciones experimentales.
\bigskip
\par
Los valores para este tipo de respuesta son los siguientes:
\begin{enumerate}
\item \textbf{100 Continúa}:Esta respuesta significa que el servidor ha recibido los encabezados de la petición, y que el cliente debería proceder a enviar el cuerpo de la misma (en el caso de peticiones para las cuales el cuerpo necesita ser enviado; por ejemplo, una petición Hypertext Transfer Protocol). Si el cuerpo de la petición es largo, es ineficiente enviarlo a un servidor, cuando la petición ha sido ya rechazada, debido a encabezados inapropiados. Para hacer que un servidor cheque si la petición podría ser aceptada basada únicamente en los encabezados de la petición, el cliente debe enviar Expect: 100-continue como un encabezado en su petición inicial (vea Plantilla:Web-RFC: Expect header) y verificar si un código de estado 100 Continue es recibido en respuesta, antes de continuar (o recibir 417 Expectation Failed y no continuar).
\item \textbf{101 Conmutando protocolos}
\item \textbf{102 Procesando (WebDAV - RFC 2518)}
\end{enumerate}

\section{Peticiones correctas(2xx)}
Esta clase de código de estado indica que la petición fue recibida correctamente, entendida y aceptada. Sus posibles valores serán los siguientes:
\bigskip
\par
\begin{enumerate}
\item \textbf{200 OK}: Respuesta estándar para peticiones correctas.
\item \textbf{201 Creado}: La petición ha sido completada y ha resultado en la creación de un nuevo recurso.
\item \textbf{202 Aceptada}: La petición ha sido aceptada para procesamiento, pero este no ha sido completado. La petición eventualmente pudiere no ser satisfecha, ya que podría ser no permitida o prohibida cuando el procesamiento tenga lugar.
\item \textbf{203 Información no autoritativa (desde HTTP/1.1)}
\item \textbf{204 Sin contenido}
\item \textbf{205 Recargar contenido.}
\item \textbf{206 Contenido parcial}: La petición servirá parcialmente el contenido solicitado. Esta característica es utilizada por herramientas de descarga como wget para continuar la transferencia de descargas anteriormente interrumpidas, o para dividir una descarga y procesar las partes simultáneamente.
\item \textbf{207 Estado múltiple (Multi-Status, WebDAV)}: El cuerpo del mensaje que sigue es un mensaje XML y puede contener algún número de códigos de respuesta separados, dependiendo de cuántas sub-peticiones sean hechas.
\end{enumerate}

\section{Redirecciones(3xx)}
El cliente tiene que tomar una acción adicional para completar la petición.
\bigskip
\par
Esta clase de código de estado indica que una acción subsecuente necesita efectuarse por el agente de usuario para completar la petición. La acción requerida puede ser llevada a cabo por el agente de usuario sin interacción con el usuario si y sólo si el método utilizado en la segunda petición es GET o HEAD. El agente de usuario no debe redirigir automáticamente una petición más de 5 veces, dado que tal funcionamiento indica usualmente un bucle infinito. Los posibles valores para este tipo de respuesta serán los siguientes:
\bigskip
\par
\begin{enumerate}
\item \textbf{300 Múltiples opciones}: Indica opciones múltiples para el URI que el cliente podría seguir. Esto podría ser utilizado, por ejemplo, para presentar distintas opciones de formato para video, listar archivos con distintas extensiones o word sense disambiguation.
\item \textbf{301 Movido permanentemente}: Esta y todas las peticiones futuras deberían ser dirigidas a la URI dada.
\item \textbf{302 Movido temporalmente}: Este es el código de redirección más popular, pero también un ejemplo de las prácticas de la industria contradiciendo el estándar. La especificación HTTP/1.0 (RFC 1945) requería que el cliente realizara una redirección temporal (la frase descriptiva original fue "Moved Temporarily"), pero los navegadores populares lo implementaron como 303 See Other. Por tanto, HTTP/1.1 añadió códigos de estado 303 y 307 para eliminar la ambigüedad entre ambos comportamientos. Sin embargo, la mayoría de aplicaciones web y bibliotecas de desarrollo aún utilizan el código de respuesta 302 como si fuera el 303.
\item \textbf{303 Vea otra (desde HTTP/1.1)}: La respuesta a la petición puede ser encontrada bajo otra URI utilizando el método GET.
\item \textbf{304 No modificado}: Indica que la petición a la URL no ha sido modificada desde que fue requerida por última vez. Típicamente, el cliente HTTP provee un encabezado como If-Modified-Since para indicar una fecha y hora contra la cual el servidor pueda comparar. El uso de este encabezado ahorra ancho de banda y reprocesamiento tanto del servidor como del cliente.
\item \textbf{305 Utilice un proxy (desde HTTP/1.1)}: Muchos clientes HTTP (como Mozilla2 e Internet Explorer) no se apegan al estándar al procesar respuestas con este código, principalmente por motivos de seguridad.
\item \textbf{306 Cambie de proxy}: Esta respuesta está descontinuada.
\item \textbf{307 Redirección temporal (desde HTTP/1.1)}: Se trata de una redirección que debería haber sido hecha con otra URI, sin embargo aún puede ser procesada con la URI proporcionada. En contraste con el código 303, el método de la petición no debería ser cambiado cuando el cliente repita la solicitud. Por ejemplo, una solicitud POST tiene que ser repetida utilizando otra petición POST.
\end{enumerate}

\section{Errores de cliente(4xx)}
La solicitud contiene sintaxis incorrecta o no puede procesarse.
\bigskip
\par
La intención de la clase de códigos de respuesta 4xx es para casos en los cuales el cliente parece haber errado la petición. Excepto cuando se responde a una petición HEAD, el servidor debe incluir una entidad que contenga una explicación a la situación de error, y si es una condición temporal o permanente. Estos códigos de estado son aplicables a cualquier método de solicitud (como GET o POST). Los agentes de usuario deben desplegar cualquier entidad al usuario. Estos son típicamente los códigos de respuesta de error más comúnmente encontrados:
\bigskip
\par
\begin{enumerate}
\item \textbf{400 Solicitud incorrecta}: La solicitud contiene sintaxis errónea y no debería repetirse.
\item \textbf{401 No autorizado}: Similar al 403 Forbidden, pero específicamente para su uso cuando la autentificación es posible pero ha fallado o aún no ha sido provista. Vea autentificación HTTP básica y Digest access authentication.
\item \textbf{402 Pago requerido}: La intención original era que este código pudiese ser usado como parte de alguna forma o esquema de Dinero electrónico o micropagos, pero eso no sucedió, y este código nunca se utilizó.
\item \textbf{403 Prohibido}: La solicitud fue legal, pero el servidor se rehúsa a responderla. En contraste a una respuesta 401 No autorizado, la autentificación no haría la diferencia.
\item \textbf{404 No encontrado}: Recurso no encontrado. Se utiliza cuando el servidor web no encuentra la página o recurso solicitado.
\item \textbf{405 Método no permitido}: Una petición fue hecha a una URI utilizando un método de solicitud no soportado por dicha URI; por ejemplo, cuando se utiliza GET en una forma que requiere que los datos sean presentados vía POST, o utilizando PUT en un recurso de sólo lectura.
\item \textbf{406 No aceptable}: El servidor no es capaz de devolver los datos en ninguno de los formatos aceptados por el cliente, indicados por éste en la cabecera "Accept" de la petición.
\item \textbf{407 Autenticación Proxy requerida}
\item \textbf{408 Tiempo de espera agotado}: El cliente falló al continuar la petición - excepto durante la ejecución de videos Adobe Flash cuando solo significa que el usuario cerró la ventana de video o se movió a otro.
\item \textbf{409 Conflicto}: Indica que la solicitud no pudo ser procesada debido a un conflicto con el estado actual del recurso que esta identifica.
\item \textbf{410 Ya no disponible}: Indica que el recurso solicitado ya no está disponible y no lo estará de nuevo. Debería ser utilizado cuando un recurso ha sido quitado de forma permanente. Si un cliente recibe este código no debería volver a solicitar el recurso en el futuro. Por ejemplo un buscador lo eliminará de sus índices y lo hará más rápidamente que utilizando un código 404.
\item \textbf{411 Requiere longitud}
\item \textbf{412 Falló precondición}
\item \textbf{413 Solicitud demasiado larga}
\item \textbf{414 URI demasiado larga}
\item \textbf{415 Tipo de medio no soportado}
\item \textbf{416 Rango solicitado no disponible}: El cliente ha preguntado por una parte de un archivo, pero el servidor no puede proporcionar esa parte, por ejemplo, si el cliente preguntó por una parte de un archivo que está más allá de los límites del fin del archivo.
\item \textbf{417 Falló expectativa}
\item \textbf{421 Hay muchas conexiones desde esta dirección de internet}
\item \textbf{422 Entidad no procesable (WebDAV - RFC 4918)}: La solicitud está bien formada pero fue imposible seguirla debido a errores semánticos.
\item \textbf{423 Bloqueado (WebDAV - RFC 4918)}: El recurso al que se está teniendo acceso está bloqueado.
\item \textbf{424 Falló dependencia (WebDAV) (RFC 4918)}: La solicitud falló debido a una falla en la solicitud previa.
\item \textbf{425 Colección sin ordenar}: Definido en los drafts de WebDav Advanced Collections, pero no está presente en "Web Distributed Authoring and Versioning (WebDAV) Ordered Collections Protocol" (RFC 3648).
\item \textbf{426 Actualización requerida (RFC 2817)}: El cliente debería cambiarse a TLS/1.0.
\item \textbf{449 Reintente con}: Una extensión de Microsoft: La petición debería ser reintentada después de hacer la acción apropiada.
\end{enumerate}

\section{Errores de servidor(5xx)}
El servidor falló al completar una solicitud aparentemente válida.
\bigskip
\par
Los códigos de respuesta que comienzan con el dígito "5" indican casos en los cuales el servidor tiene registrado aún antes de servir la solicitud, que está errado o es incapaz de ejecutar la petición. Excepto cuando está respondiendo a un método HEAD, el servidor debe incluir una entidad que contenga una explicación de la situación de error, y si es una condición temporal o permanente. Los agentes de usuario deben desplegar cualquier entidad incluida al usuario. Estos códigos de repuesta son aplicables a cualquier método de petición.
\bigskip
\par
\begin{enumerate}
\item \textbf{500 Error interno}: Es un código comúnmente emitido por aplicaciones empotradas en servidores web, mismas que generan contenido dinámicamente, por ejemplo aplicaciones montadas en IIS o Tomcat, cuando se encuentran con situaciones de error ajenas a la naturaleza del servidor web.
\item \textbf{501 No implementado}
\item \textbf{502 Pasarela incorrecta}
\item \textbf{503 Servicio no disponible}
\item \textbf{504 Tiempo de espera de la pasarela agotado}
\item \textbf{505 Versión de HTTP no soportada}
\item \textbf{506 Variante también negocia (RFC 2295)}
\item \textbf{507 Almacenamiento insuficiente (WebDAV - RFC 4918)}
\item \textbf{509 Límite de ancho de banda excedido}: Este código de estatus, a pesar de ser utilizado por muchos servidores, no es oficial.
\item \textbf{510 No extendido (RFC 2774)}
\end{enumerate}
