% Formato para un capítulo cualquiera

%Título del capítulo
\chapter{Base de Datos SocialNetwork} 

\section{Introducción}
Para la realización del proyecto SNSAngelGuard se ha requerido el diseño de una base de datos que persista todos los elementos necesarios para el análisis y el procesado de la información. Como se ha indicado en los capítulos anteriores, la base de datos se diseño para albergar datos no sólo de Facebook, sino también de otras redes sociales, como pueden ser Tuenti o Twitter, bajo el estándar de red social de Open Social.
\bigskip
\par
Además, posteriormente, se requirió de una estructura que pudiera controlar la ejecución de la aplicación SNSAngelGuard, enlazando el concepto de usuario de la aplicación con el usuario de una red social, por lo que, un usuario que ejecuta la aplicación debe tener un usuario válido en la red social en la que lo esté ejecutando. Esta estructura es la que almacena toda la información referente a la configuración del usuario en la aplicación, abarcando desde los ángeles que define, hasta la periodicidad con que se ejecutarán sus filtros.
\bigskip
\par
En el diseño de la base de datos se aprecian los siguientes módulos funcionales:
\begin{enumerate}
\item Módulo funcional de Configuración: Este módulo contendrá todas aquellas tablas necesarias para almacenar la configuración del usuario referente a filtros en ejecución, periodicidad de éstos y ángeles seleccionados para cada filtro. Esta información podrá cambiar cada vez que el usuario entre en la aplicación y guarde una información distinta a la que había originalmente.
\item Módulo funcional de Datos Comunes: Este módulo contendrá aquellas estructuras de información comunes a Facebook y OpenSocial reagrupadas, para dotar de mayor estabilidad a la base de datos reordenando aquella información que pueda ser simplificada en ambas redes.
\item Módulo funcional de OpenSocial: Este módulo contendrá todas aquellas tablas necesarias para guardar la información referente a OpenSocial. Contendrá funcionalmente las tablas para guardar la misma información que las estructuras definidas para almacenar información de Facebook. Aunque en éste desarrollo no se va a utilizar, se explicará el modelo de datos para sucesivos desarrollos. 
\item Módulo funcional de Facebook: Contendrá todas aquellas tablas necesarias para albergar información personal del usuario de Facebook que actualmente utiliza la aplicación. Se podrán guardar datos personales de cualquier tipo, desde datos personales hasta referencias a los circulo familiares a los que pertenece o enlaces a las fotografías que cuelga en su perfil.
\end{enumerate}
\bigskip
\par
Enumerados los diferentes bloques funcionales, pasaremos a definirlos formalmente en los siguientes apartados.

\section{Módulo funcional de Configuración}
Este módulo, como bien se ha especificado anteriormente, contendrá todas las tablas relacionadas con la configuración de la propia aplicación. Será un modelo sencillo en el que inicialmente se puede guardar la siguiente información:
\begin{enumerate}
\item Datos del usuario: Será la tabla maestra que controlará toda la aplicación. Todo usuario de la aplicación deberá tener un registro persistido en ésta tabla, aunque en el resto de tablas no contenga datos. En ella se almacenarán datos tales como su correo electrónico, el indicador de si la aplicación está activa, la fecha de la ultima actualización de datos, ..., etc.
\item Ángeles seleccionados: Será la tabla que almacene todos los datos de los ángeles que han sido definidos para controlar al usuario de la aplicación. Esta tabla estará relacionada con las tablas de filtros, siendo éstas últimas las que irán a buscar datos de los ángeles configurados en los filtros a la primera.
\item Filtros configurados: Actualmente existen cuatro filtros disponibles:
\begin{enumerate}
\item Filtro de control vocabulario ofensivo.
\item Filtro de control de amigos.
\item Filtro de control de privacidad.
\item Filtro de control de visitas.
\end{enumerate}
\end{enumerate}
Para todos ellos existirá una tabla en la que se almacenará si el filtro está activo y la frecuencia con la que se desea aplicar sobre la información. Cada tabla tendrá una relación con la tabla de ángeles en la que se especificará los ángeles que pertenecen a cada filtro.
\bigskip
\par 
Funcionalmente, el módulo ya está definido. Pasemos ahora a explicar detalladamente el modelo físico que se ha diseñado para tal fin.
\subsection{Tabla user\_settings}
Esta tabla albergará la información del usuario. Será la tabla maestra de la aplicación, por lo que todo usuario que pertenezca a la aplicación deberá tener un registro en ésta. Se relacionará con la tabla user en una relación 1 a 1, por lo que se cumple la anterior premisa, ya que ésta última contendrá datos maestros tanto de Facebook como de OpenSocial. Los campos de la tabla user\_settings son los mostrados en la tabla \ref{tabUserSettings}.
\begin{table}
\begin{center}

\begin{tabular}[c]{| l | l | p{60mm} |}\hline
\textbf{Campo}&\textbf{Tipo}&\textbf{Descripción} \\ \hline
uid & VARCHAR(100) & Contiene el identificador del usuario en la aplicación. Corresponderá al usuario de Facebook. \\ \hline
user\_name & VARCHAR(100) & Nombre del usuario \\ \hline
user\_email & VARCHAR(200) & Email del usuario \\ \hline
legal\_accepted & VARCHAR(1) & Indicador de aceptación del acuerdo legal por parte del usuario. \\ \hline
last\_check & TIMESTAMP & Fecha de la última ejecución de la aplicación cliente. \\ \hline
uid\_public & VARCHAR(200) & Identificador público del usuario. \\ \hline
app\_activated & VARCHAR(1) & Indica si la aplicación está activada o no. \\ \hline
user\_session & VARCHAR(200) & Token de sesión de Facebook para accesos offline. \\ \hline
locale\_settings\_id\_locale & VARCHAR(100) & Índice de la tabla locale\_settings que indica el idioma en el que se va a mostrar la aplicación. \\ \hline
bakup\_check & TIMESTAMP & Indica la última actualización de información del usuario. \\ \hline
\end{tabular}
\end{center}
\caption{Tabla user\_settings} \label{tabUserSettings}
\end{table}

\subsection{Tabla locale\_settings}
Esta tabla almacenará toda la información referente al idioma en que el usuario ejecuta la aplicación. Contendrá la información necesaria para poder traducir la aplicación(títulos de páginas, botones, ayuda, ..., etc.) al idioma en el que el usuario tenga configurado Facebook, es decir, si su perfil de dicha Red Social está en Castellano, la aplicación se mostrará en Castellano. Si por el contrario, estuviera en Inglés, la aplicación se mostraría en Inglés. Actualmente, la aplicación SNSAngelGuard cuenta con soporte para Castellano e Inglés. 
\bigskip
\par
Esta tabla estará relacionada mediante el campo locale\_settings\_id\_locale perteneciente a la tabla user\_settings. Éste campo contendrá el identificador del registro de la tabla locale\_settings a la cual hace referencia, obteniendo a partir de éste todos los recursos de idioma de la aplicación. La tabla \ref{tabLocaleSettings} muestra toda la información que contiene ésta estructura.

\begin{center}
\begin{longtable}{|l|l|p{60mm} |}

%Cabecera y primera hoja de la tabla
\caption{Tabla locale\_settings} \label{tabLocaleSettings}\\
\hline \multicolumn{1}{|l|}{\textbf{Campo}} & \multicolumn{1}{l|}{\textbf{Tipo}} & \multicolumn{1}{p{60mm} |}{\textbf{Descripción}} \\ 
\hline 
\endfirsthead

%Cabecera y resto de hojas de la tabla
\hline \multicolumn{1}{|l|}{\textbf{Campo}} & \multicolumn{1}{l|}{\textbf{Tipo}} & \multicolumn{1}{p{60mm}|}{\textbf{Descripción}} \\ \hline 
\endhead

\hline
id\_locale & VARCHAR(100) & Identificador único de la tabla. \\ \hline
acceptingTerms & VARCHAR(1000) & Cabecera del acuerdo legal. \\ \hline
legalAccepted & MEDIUMTEXT & Texto de aceptación de los términos de ejecución de la aplicación. \\ \hline
btnAgreeAT & VARCHAR(100) & Título del botón `Aceptar' de la página de Aceptación de Términos. \\ \hline
btnCancelAT & VARCHAR(100) & Título del botón `Cancelar' de la página de Aceptación de Términos. \\ \hline
titleAT & VARCHAR(100) & Título de la página de Aceptación de Términos. \\ \hline
titleSettings & VARCHAR(100) & Título de la página contenedora de la aplicación. \\ \hline
titleMenSettings & VARCHAR(100) & Título de las pestañas de la página contenedora. \\ \hline
btnSaveCheckSettings & VARCHAR(100) & Título del botón `Guardar' de la página de contenedora de la aplicación. \\ \hline
titleSettAng & VARCHAR(100) & Título principal de la página de la pestaña `Angeles' \\ \hline
titleFriendsContentSettAng & VARCHAR(100) & Título del frame de selección de ángeles de Facebook \\ \hline
titleFriendsImportSettAng & VARCHAR(100) & Títulos de los frams de selección de contactos de Google u otros contactos\\ \hline
txtNameTutorSettAng & VARCHAR(100) & Título para el nombre de un contacto del frame `Otros contactos'. \\ \hline
txtEmailTutorSettAng & VARCHAR(100) & Título para el email de un contacto del frame `Otros contactos'. \\ \hline
btnImportSettAng & VARCHAR(100) & Título del botón "Importar" contactos de Google. \\ \hline
titleFbListSettAng & VARCHAR(100) &  Contiene los nombres de las pestañas que visualizan los contactos de Facebook. \\ \hline
subTitleAngelSettAng & VARCHAR(100) &  Contiene los subtitulos que aparecen en cada contacto de Facebook. \\ \hline
titleSettVig & VARCHAR(100) & Título de la página de configuración de los vigilantes. \\ \hline
titleVigilantSettVig & VARCHAR(100) & Titulo del frame `Vigilantes' de la pestaña `Vigilantes'. \\ \hline
titleVigSettVig & VARCHAR(100) & Título de los filtros vigilantes disponibles.  \\ \hline
titleVigDescriptionSettVig & VARCHAR(100) &  Descripciones de cada vigilante. \\ \hline
titleVigFrecSettVig & VARCHAR(100) &  Título de la lista desplegable `Frecuencia'. \\ \hline
titleVigFrecSelectSettVig & VARCHAR(100) &  Valores disponibles de la lista desplegable `Frecuencia'. \\ \hline
titleVigAngSettVig & VARCHAR(100) & Titulo del frame `Angeles' de la pestaña `Vigilantes'. \\ \hline
titleAngConfirm & VARCHAR(100) &  Título del email de confirmación enviado a un ángel. \\ \hline
desInfoAngConfirm & VARCHAR(100) & Cuerpo del email de confirmación enviado a un ángel. \\ \hline
aceptConfAngConfirm & VARCHAR(100) &  Título del link de confirmación enviado en el email a un ángel. \\ \hline
cancelConfAngConfirm & VARCHAR(100) &  Título del link de cancelación enviado en el email a un ánge. \\ \hline
infoAngGuard & VARCHAR(100) &  Información acerca de la aplicación eviado en cada email. \\ \hline
titleAngUser & VARCHAR(100) &  Título de la página de información acerca del usuario que pide confirmación a un ángel. \\ \hline
nameUserAngUser & VARCHAR(100) & Título del nombre del usuario de Facebook. \\ \hline
btnCloseAngUser & VARCHAR(100) &  Título del botón `Cerrar' de la página de información de un usuario de Facebook. \\ \hline
titleGoogleCont & VARCHAR(100) &  Titulo de la página de importación de contactos de Google. \\ \hline
titleContGoogleCont & VARCHAR(100) &  Subtítulo de la página de importación de contactos de Google. \\ \hline
btnLogGoogleCont & VARCHAR(100) &  Título del botón de login en Google. \\ \hline
titleNameContactGoogleCont & VARCHAR(100) &  Nombre del contacto de Google.\\ \hline
titleEmailContactGoogleCont & VARCHAR(100) &  Email del contacto de Google. \\ \hline
btnAceptGoogleCont & VARCHAR(100) & Título del botón `Aceptar' de la página de importación de contactos de Google. \\ \hline
btnCancelGoogleCont & VARCHAR(100) & Título del botón `Cancelar' de la página de importación de contactos de Google. \\ \hline
helpMe & MEDIUMTEXT &  Contenido de la página de ayuda de la aplicación. \\ \hline
warnings & MEDIUMTEXT &  Contenido de los mensajes de aviso y errores que pueden producirse en la aplicación. \\ \hline
titleInformationMessage & VARCHAR(100) & Título para los mensajes de aviso y errores. \\ \hline
informationMessage & MEDIUMTEXT &  Contenido de las páginas de confirmación offline para los ángeles. \\ \hline
mailDelete & MEDIUMTEXT &  Contenido del email de eliminación de un ángel por parte de un usuario. \\ \hline
mailNotification & MEDIUMTEXT &  Contenido del email de notificación de actividad de un usuario hacia sus ángeles. \\ \hline
altContactsAngelsEd & MEDIUMTEXT &  Títulos para los botones de acción para otros contactos.\\ \hline 
titleVisitsFilterOptions & MEDIUMTEXT &  Títulos para las diferentes estadísticas utilizadas en el filtro de visitas.\\ \hline 

\end{longtable}
\end{center}

\subsection{Tabla settings\_angels}
Esta tabla contendrá la información de todos aquellos ángeles definidos para cada usuario. Estará relacionada con la tabla user\_settings con una relación N a N, en la que muchos usuarios podrán tener muchos ángeles. De cada ángel se obtiene la siguiente información:
\begin{enumerate}
\item Su identificador. En éste caso, corresponderá con su correo electrónico para el envío de notificaciones.
\item Su dirección de correo electrónico.
\item Su nombre.
\item El tipo de angel, es decir, si pertenece a Facebook, Google u otros contactos.
\item Si ha aceptado ser el ángel de algún usuario de la aplicación.
\item El usuario al que pertenece dentro de la aplicación.
\item La imagen del ángel. Si pertenece a Facebook, se almacenará su foto de perfil en el sistema. En otro caso, el sistema almacenará la imagen de perfil por defecto.
\end{enumerate}
\bigskip
\par
Toda la información anterior puede verse reflejada en la tabla \ref{tabSettingsAngels}.
\begin{table}
\begin{center}

\begin{tabular}[c]{| l | l | p{60mm} |}\hline
\textbf{Campo}&\textbf{Tipo}&\textbf{Descripción} \\ \hline
uid\_angel & INTEGER & Identificador único del ángel en la tabla. \\ \hline
id\_angel & VARCHAR(100) & Id del ángel, es decir, su email. \\ \hline
name\_angel & VARCHAR(200) & Nombre del ángel. \\ \hline
img\_angel & VARCHAR(200) & Imagen del ángel. \\ \hline
type\_angel & VARCHAR(10) & Tipo del ángel. \\ \hline
accept\_angel & VARCHAR(1) & Indica si el ángel ha aceptado seguir a un usuario o no. Se representará con el valor 1 en caso afirmativo y 0 en caso negativo. \\ \hline
user\_prop\_angel & VARCHAR(100) & Usuario de la aplicación propietario del ángel. \\ \hline
confirm\_angel & VARCHAR(1) & Indica si el ángel ha respondido a la solicitud. Se representará con el valor 1 en caso afirmativo y 0 en caso negativo. \\ \hline
\end{tabular}
\end{center}
\caption{Tabla settings\_angels} \label{tabSettingsAngels}
\end{table}

\subsection{Tabla settings\_fltWall}
Para guardar la configuración del filtro de control de lenguaje para un determinado usuario de la aplicación, se diseñó la tabla settings\_fltWall. Para cada usuario, se almacenará la siguiente información:
\begin{enumerate}
\item Frecuencia a la que se ejecutará el filtro de control de lenguaje. Este campo contendrá un valor entre el 0 y el 6 que indicará lo siguiente:
\begin{enumerate}
\item 0: El filtro se ejecutará diariamente.
\item 1: El filtro se ejecutará cada semana.
\item 2: El filtro se ejecutará cada dos semanas.
\item 3: El filtro se ejecutará mensualmente(valor por defecto).
\item 4: El filtro se ejecutará cada dos meses.
\item 5: El filtro se ejecutará cada seis meses.
\item 6: El filtro se ejecutará anualmente.
\end{enumerate}
\item Si se encuentra activo.
\item La fecha de la última vez que se ejecutó el filtro.
\end{enumerate}
\bigskip
\par
Esta tabla, además, estará relacionada N a N con la tabla settings\_angels, de la cual se obtendrán aquellos ángeles que estén configurados para este filtro, a los cuales se les enviará, cuando la frecuencia indique, los informes de notificación correspondientes a la ejecución del filtro.
\bigskip
\par
La estructura física de la tabla puede observarse en la tabla \ref{tabSettingsFltWall}.
\begin{table}
\begin{center}
\begin{tabular}[c]{| l | l | p{60mm} |}\hline
\textbf{Campo}&\textbf{Tipo}&\textbf{Descripción} \\ \hline
user\_settings\_uid & VARCHAR(1000) & Identificador único del usuario de la aplicación. \\ \hline
frec\_fltWall & VARCHAR(100) & Frecuencia a la que se ejecutará el filtro de control de lenguaje. \\ \hline
active\_fltWall & VARCHAR(1) & Indicador de actividad del filtro. El valor 1 indicará que el filtro se encuentra activo y el valor 0 indicará que no se encuentra activado. \\ \hline
last\_check & TIMESTAMP & Fecha de la última ejecución del filtro. \\ \hline
\end{tabular}
\end{center}
\caption{Tabla settings\_fltWall} \label{tabSettingsFltWall}
\end{table}

\subsection{Tabla settings\_fltFriends}
Esta tabla contendrá toda la información necesaria para la configuración del filtro de control de amistades. Para cada usuario se almacenará la siguiente información:
\begin{enumerate}
\item Frecuencia a la que se ejecutará el filtro. Como la tabla anterior, este campo almacenará los mismos códigos de frecuencia.
\item Si se encuentra activo.
\item La fecha de la última vez que se ejecutó el filtro.
\end{enumerate}
Para almacenar los ángeles con los que contará el filtro, se establece una relación N a N con la tabla settings\_angels, a los cuales se les enviará los informes de notificación de la ejecución del filtro.
\bigskip
\par
La estructura física de la tabla puede observarse en la tabla \ref{tabSettingsFltFriends}.
\begin{table}
\begin{center}
\begin{tabular}[c]{| l | l | p{60mm} |}\hline
\textbf{Campo}&\textbf{Tipo}&\textbf{Descripción} \\ \hline
user\_settings\_uid & VARCHAR(1000) & Identificador único del usuario de la aplicación. \\ \hline
frec\_fltFriends & VARCHAR(100) & Frecuencia a la que se ejecutará el filtro de control de amistades. \\ \hline
active\_fltFriends & VARCHAR(1) & Indicador de actividad del filtro. El valor 1 indicará que el filtro se encuentra activo y el valor 0 indicará que no se encuentra activado. \\ \hline
last\_check & TIMESTAMP & Fecha de la última ejecución del filtro. \\ \hline
\end{tabular}
\end{center}
\caption{Tabla settings\_fltFriends} \label{tabSettingsFltFriends}
\end{table}

\subsection{Tabla settings\_fltVist}
Esta tabla contendrá toda la información necesaria para la ejecución del filtro de visitas de un determinado usuario. Para cada uno de éstos se almacenará la siguiente información:
\begin{enumerate}
\item Frecuencia a la que se almacenará el filtro. Con el mismo código de valores que los filtros anteriores.
\item Si se encuentra activo para el usuario propietario del filtro.
\item La fecha de la última vez que se ejecutó el filtro.
\end{enumerate}
Para almacenar los ángeles con los que contará el filtro, se establece una relación N a N con la tabla settings\_angels, a los cuales se les enviará los informes de notificación de la ejecución del filtro.
\bigskip
\par
La estructura física de la tabla puede observarse en la tabla \ref{tabSettingsFltVist}.
\begin{table}
\begin{center}
\begin{tabular}[c]{| l | l | p{60mm} |}\hline
\textbf{Campo}&\textbf{Tipo}&\textbf{Descripción} \\ \hline
user\_settings\_uid & VARCHAR(1000) & Identificador único del usuario de la aplicación. \\ \hline
frec\_fltVist & VARCHAR(100) & Frecuencia a la que se ejecutará el filtro de control de visitas. \\ \hline
active\_fltVist & VARCHAR(1) & Indicador de actividad del filtro. El valor 1 indicará que el filtro se encuentra activo y el valor 0 indicará que no se encuentra activado. \\ \hline
last\_check & TIMESTAMP & Fecha de la última ejecución del filtro. \\ \hline
\end{tabular}
\end{center}
\caption{Tabla settings\_fltVist} \label{tabSettingsFltVist}
\end{table}

\subsection{Tabla settings\_fltPriv}
Esta tabla contendrá toda la información necesaria para la ejecución del filtro de privacidad de un determinado usuario. Para cada uno de éstos se almacenará la siguiente información:
\begin{enumerate}
\item Frecuencia a la que se almacenará el filtro. Con el mismo código de valores que los filtros anteriores.
\item Si se encuentra activo para el usuario propietario del filtro.
\item La fecha de la última vez que se ejecutó el filtro.
\end{enumerate}
Para almacenar los ángeles con los que contará el filtro, se establece una relación N a N con la tabla settings\_angels, a los cuales se les enviará los informes de notificación de la ejecución del filtro.
\bigskip
\par
La estructura física de la tabla puede observarse en la tabla \ref{tabSettingsFltPriv}.
\begin{table}
\begin{center}
\begin{tabular}[c]{| l | l | p{60mm} |}\hline
\textbf{Campo}&\textbf{Tipo}&\textbf{Descripción} \\ \hline
user\_settings\_uid & VARCHAR(1000) & Identificador único del usuario de la aplicación. \\ \hline
frec\_fltPriv & VARCHAR(100) & Frecuencia a la que se ejecutará el filtro de control de privacidad. \\ \hline
active\_fltPriv & VARCHAR(1) & Indicador de actividad del filtro. El valor 1 indicará que el filtro se encuentra activo y el valor 0 indicará que no se encuentra activado. \\ \hline
last\_check & TIMESTAMP & Fecha de la última ejecución del filtro. \\ \hline
\end{tabular}
\end{center}
\caption{Tabla settings\_fltPriv} \label{tabSettingsFltPriv}
\end{table}

\section{Módulo funcional de Datos Comunes}
En éste modulo se agrupará toda aquella información que pueda ser común entre las Redes Sociales a analizar, tales como Facebook y OpenSocial. Evidentemente, los datos más propensos a ser reagrupados serán aquellos que atañen directamente a la faceta personal del usuario, tales como sus datos personales, sus intereses, relaciones personales, estatus, citas sobre libros, etc. Por esta razón, se ha diseñado una tabla maestra denominada `user' que será la que agrupe todos estos datos.

\subsection{Tabla user}
Esta tabla será maestra, es decir, estará relacionada directamente con el nivel superior, es decir, con la tabla user\_settings y con las del nivel inferior, user\_facebook y user\_opensocial. Es evidente que un usuario de la aplicación debe tener una entrada en ésta tabla para que los datos sean consistentes. Los datos que contiene son de índole personal y estarán almacenados físicamente en la estructura mostrada en la tabla \ref{tabUser}.
\bigskip
\par
\begin{table}
\begin{center}
\begin{tabular}[c]{| l | l | p{60mm} |}\hline
\textbf{Campo}&\textbf{Tipo}&\textbf{Descripción} \\ \hline
user\_settings\_uid & VARCHAR(1000) & Identificador único del usuario de la aplicación. \\ \hline
SEX & VARCHAR(45) & Sexo del usuario de la aplicación. \\ \hline
RELIGION & VARCHAR(450) & Tendencias religiosas del usuario. \\ \hline
RELATIONSHIP\_STATUS & VARCHAR(450) & Estado sentimental. \\ \hline
POLITICAL & VARCHAR(450) & Tendencias politicas.\\ \hline
ACTIVITIES & VARCHAR(450) & Actividades que realiza el usuario. \\ \hline
INTERESTS & VARCHAR(450) & Intereses del usuario. \\ \hline
IS\_APP\_USER & TINYINT & Indicador del perfil activo dentro de la Red Social. \\ \hline
MUSIC & VARCHAR(450) & Tendencias musicales del usuario. \\ \hline
TV & VARCHAR(450) & Programas televisivos que sigue o ha seguido actualmente. \\ \hline
MOVIES & VARCHAR(450) & Películas favoritas. \\ \hline
BOOKS & VARCHAR(450) & Libros favoritos. \\ \hline
ABOUT\_ME & VARCHAR(450) & Descripción propia del usuario dentro de su perfil en la Red Social. \\ \hline
STATUS & VARCHAR(450) & Estatus actual del usuario. \\ \hline
QUOTES & VARCHAR(450) & Citas o frases famosas que definen a un usuario. \\ \hline
user\_facebook\_id\_user\_facebook & VARCHAR(100) & Identificador de su usuario en Facebook. \\ \hline
user\_openSocial\_id\_user\_openSocial & VARCHAR(100) & Identificador de su usuario en OpenSocial. \\ \hline
\end{tabular}
\end{center}
\caption{Tabla user} \label{tabUser}
\end{table}

\section{Módulo funcional de OpenSocial}
Aunque como se ha especificado anteriormente no se ha realizado funcionalidad alguna para la Red Social OpenSocial, si que se ha preparado la base de datos para un próximo desarrollo sobre dicha Red Social que conlleve el análisis y estudio de los datos que se puedan generar. Para ello, se pasará a describir las estructuras físicas diseñadas para ser utilizadas a corto-medio plazo.

\subsection{Tabla user\_openSocial}
Es la tabla maestra de los datos de OpenSocial. Como se ha comentado anteriormente, desciende de la tabla de datos comunes \textbf{user}. En ella se centrarán todas las relaciones de datos e irán gestionadas desde esta tabla. Estas relaciones serán las siguientes:
\begin{enumerate}
\item Sus números de teléfono irán gestionados siguiendo una relación 1 a N con la tabla \textbf{Phones\_openSocial}.
\item Si es fumador, la tabla almacenará un código que será una entrada a la tabla maestra \textbf{Smoker\_openSocial}.
\item Sus preferencias en cuanto a tendencias sociales seguirán una relación 1 a N con la tabla \textbf{LookingFor\_openSocial}.
\item Todas sus direcciones de email se almacenarán, mediante una relación 1 a N, con la tabla \textbf{Email\_openSocial}.
\item Su sexo irá descrito mediante un codigo que referenciará una entrada a la tabla maestra \textbf{Gender\_openSocial}.
\item Su disponibilidad en OpenSocial irá descrita mediante un código que referenciará una entrada a la tabla maestra \textbf{Presence\_openSocial}.
\item Su grado de alcoholismo irá descrita mediante un código que referenciará una entrada a la tabla maestra \textbf{Drinker\_openSocial}.
\item Sus direcciones particulares seguirán una relación 1 a N con la tabla \textbf{Address\_ openSocial} que a su vez estará relacionada con la tabla \textbf{Organization\_ openSocial}, la cual será capaz de almacenar los lugares de trabajo o sitios donde haya estudiado mediante relaciones N a N.
\item Sus amigos dentro de OpenSocial serán almacenados, mediante una relación N a N, con la tabla \textbf{Friends\_openSocial}.
\item Las URLs sobre videos, clips de audio o páginas de interés serán almacenados, mediante una relación N a N, con la tabla \textbf{url\_openSocial}.
\item Los diferentes nombres o apodos que el usuario vaya utilizando a lo largo de su ciclo de vida en OpenSocial se almacenarán, mediante una relación N a N, con la tabla \textbf{Name\_openSocial}.
\item Los mensajes de muro que el usuario recibe en su perfil de OpenSocial se almacenarán, mediante una relación 1 a N, en la tabla \textbf{Message\_openSocial}.
\end{enumerate}


Su estructura física está representada en la tabla \ref{tabUserOpenSocial}.

\begin{center}
\begin{longtable}{|p{70mm} |l|p{60mm} |}

%Cabecera y primera hoja de la tabla
\caption{Tabla user\_openSocial} \label{tabUserOpenSocial}\\
\hline \multicolumn{1}{|l|}{\textbf{Campo}} & \multicolumn{1}{l|}{\textbf{Tipo}} & \multicolumn{1}{p{60mm} |}{\textbf{Descripción}} \\ 
\hline 
\endfirsthead

%Cabecera y resto de hojas de la tabla
\hline \multicolumn{1}{|l|}{\textbf{Campo}} & \multicolumn{1}{l|}{\textbf{Tipo}} & \multicolumn{1}{p{60mm}|}{\textbf{Descripción}} \\ \hline 
\endhead

\hline
id\_user\_openSocial & VARCHAR(100) & Identificador único de la tabla. \\ \hline
AGE & INTEGER & Edad del usuario. \\ \hline
BODYTYPE\_BUILD & VARCHAR(100)  & Complexión del usuario. \\ \hline
BODYTYPE\_EYE\_COLOR & VARCHAR(45) & Color de ojos. \\ \hline
BODYTYPE\_HAIR\_COLOR & VARCHAR(45) & Color de pelo. \\ \hline
BODYTYPE\_HEIGHT & VARCHAR(45) & Altura. \\ \hline
BODYTYPE\_WEIGHT & VARCHAR(45) & Peso. \\ \hline
CARS & VARCHAR(450) & Tendencias sobre coches. \\ \hline
CHILDREN & VARCHAR(450) & Hijos del usuario. \\ \hline
Address\_openSocial\_CURRENT\_ LOCATION & VARCHAR(10) & Dirección actual. \\ \hline
DATE\_OF\_BIRTH & DATE & Fecha de nacimiento. \\ \hline
Drinker\_openSocial\_id\_Drinker\_ openSocial & INTEGER & Grado de alcoholismo. \\ \hline
ETHNICITY & VARCHAR(450) & Raza étnica. \\ \hline
FASHION & VARCHAR(450) & Tendencias sobre moda. \\ \hline
FOOD & VARCHAR(450) & Tendencias gastronómicas. \\ \hline
Gender\_openSocial\_id\_Gender\_ openSocial & INTEGER &  Sexo del usuario. \\ \hline
HAPPIEST\_WHEN & VARCHAR(450) &  Actividades favoritas. \\ \hline
HEROES & VARCHAR(450) & Heroes. \\ \hline
HUMOR & VARCHAR(450) & Tendencias humorísticas. \\ \hline
JOB\_INTERESTS & VARCHAR(450) & Actividad laboral.  \\ \hline
LANGUAGES\_SPOKEN & VARCHAR(450) &  Idiomas hablados por el usuario. \\ \hline
LIVING\_ARRANGEMENT & VARCHAR(450) &  Alojamiento. \\ \hline
LookingFor\_openSocial\_id\_Looking For\_openSocial & INTEGER &  Relaciones sociales sobre las que está interesado. \\ \hline
Presence\_openSocial\_id\_Presence\_ openSocial & INTEGER & Presencia actual en la red. \\ \hline
NICKNAME & VARCHAR(100) &  Apodo. \\ \hline
PETS & VARCHAR(450) & Mascotas. \\ \hline
PROFILE\_URL & VARCHAR(100) &  URL de la foto de perfil del usuario. \\ \hline
ROMANCE & VARCHAR(100) &  Relación actual. \\ \hline
SCARED\_OF & VARCHAR(450) &  Miedos o temores. \\ \hline
Smoker\_openSocial\_id\_Smoker\_ openSocial & INTEGER &  Fumador. \\ \hline
SPORTS & VARCHAR(450) & Actividades deportivas del usuario. \\ \hline
TAGS & VARCHAR(450) &  Etiquetas asociadas al usuario. \\ \hline
THUMBNAIL\_URL & VARCHAR(100) &  URL a la imagen thumbnail del usuario. \\ \hline
TIME\_ZONE & DATE &  Franja horaria en la que se encuentra el usuario. \\ \hline
TURNS\_OFFS & VARCHAR(450) &  Tendencias que desagradan al usuario. \\ \hline
TURNS\_ONS & VARCHAR(450) &  Tendencias que sorprenden al usuario.\\ \hline
URLS & VARCHAR(450) &  URLs de interes del usuario. \\ \hline

\end{longtable}
\end{center}

\subsection{Tabla Phones\_openSocial}
Como se ha especificado anteriormente, será la tabla que contenga todos los números de teléfono del usuario. Conformará una relación 1 a N con la tabla \textbf{user\_openSocial}, ya que un usuario podrá tener N números de teléfono. Su estructura física estará representada en la tabla \ref{tabPhonesOpenSocial}.
\bigskip
\par
\begin{table}[h]
\begin{center}
\begin{tabular}{| l | l | p{60mm} |}\hline
\textbf{Campo}&\textbf{Tipo}&\textbf{Descripción} \\ \hline
id\_Phones\_openSocial & INTEGER & Identificador único de la tabla. \\ \hline
NUMBER & VARCHAR(45) & Numero de teléfono asociado al usuario. \\ \hline
TYPE & VARCHAR(45) & Tipo de teléfono( fijo, móvil, etc.). \\ \hline
user\_openSocial\_id\_user\_openSocial & VARCHAR(100) & Identificador de su usuario en OpenSocial. \\ \hline
\end{tabular}
\end{center}
\caption{Tabla Phones\_openSocial} \label{tabPhonesOpenSocial}
\end{table}

\subsection{Tabla Smoker\_openSocial}
Esta tabla contendrá todos los posibles valores para describir el grado de afición al tabaco por parte de un usuario. Será una tabla maestra que irá referenciada a la tabla \textbf{user\_openSocial} mediante un codigo de entrada que podrá representar los siguientes valores:
\begin{enumerate}
\item Valor \textbf{vacío}: Indicará que el usuario no ha definido esta característica en su perfil.
\item Valor \textbf{HEAVILY}: Indica que es fumador en exceso.
\item Valor \textbf{NO}: Indica que el usuario no es fumador.
\item Valor \textbf{OCCASIONALLY}: Indica que el usuario no es fumador pero puede fumar ocasionalmente.
\item Valor \textbf{QUIT}: Indica que el usuario era fumador pero actualmente lo ha dejado.
\item Valor \textbf{QUITTING}: Indica que el usuario está dejando de fumar.
\item Valor \textbf{REGULARLY}: Indica que el usuario fuma de forma regular.
\item Valor \textbf{SOCIALLY}: Indica que el usuario fuma unicamente en reuniones sociales.
\item Valor \textbf{YES}: Indica que el usuario es fumador.
\end{enumerate}
\bigskip
\par
La tabla \ref{tabSmokerOpenSocial} mostrará su organización física.
\bigskip
\par
\begin{table}[h]
\begin{center}
\begin{tabular}{| l | l | p{60mm} |}\hline
\textbf{Campo}&\textbf{Tipo}&\textbf{Descripción} \\ \hline
id\_Smoker\_openSocial & INTEGER & Identificador único de la tabla. \\ \hline
DESCRIPTION & VARCHAR(45) & Descripción para cada uno de los valores anteriores. \\ \hline
\end{tabular}
\end{center}
\caption{Tabla Smoker\_openSocial} \label{tabSmokerOpenSocial}
\end{table}

\subsection{Tabla LookingFor\_openSocial}
Esta tabla contendrá todos los valores posibles que se encuadran en la búsqueda de actividades o perfiles sociales que el usuario busca para poder relacionarse. Al igual que la tabla \textbf{Smoker\_openSocial}, será una tabla maestra cuyos valores son los siguientes:
\begin{enumerate}
\item Valor \textbf{vacío}: Indicará que el usuario no ha definido esta característica en su perfil.
\item Valor \textbf{ACTIVITY\_PARTNERS}: Compañeros de actividad que comparten los mismos intereses sociales.
\item Valor \textbf{DATING}: Amigos para un servicio a corto plazo.
\item Valor \textbf{FRIENDS}: Amigos permanentes.
\item Valor \textbf{NETWORKING}: Compañeros de actividad en internet, como jugadores en red o compañeros de trabajo.
\item Valor \textbf{RANDOM}: Aleatorio, indiferente.
\item Valor \textbf{RELATIONSHIP}: Una relación, ya sea familiar o un vínculo sentimental.
\end{enumerate}
\bigskip
\par
La tabla \ref{tabLookingForOpenSocial} mostrará su estructura física dentro del sistema.
\bigskip
\par
\begin{table}[h]
\begin{center}
\begin{tabular}{| l | l | p{60mm} |}\hline
\textbf{Campo}&\textbf{Tipo}&\textbf{Descripción} \\ \hline
id\_LookingFor\_openSocial & INTEGER & Identificador único de la tabla. \\ \hline
DESCRIPTION & VARCHAR(45) & Descripción para cada uno de los valores anteriores. \\ \hline
\end{tabular}
\end{center}
\caption{Tabla LookingFor\_openSocial} \label{tabLookingForOpenSocial}
\end{table}

\subsection{Tabla Email\_openSocial}
Esta tabla contendrá todas las direcciones de correo electrónico que hayan sido introducidas por el usuario de OpenSocial en su perfil. Estará relacionada 1 a N con la tabla \textbf{user\_openSocial} y su estructura física se muestra en la tabla \ref{tabEmailOpenSocial}.
\bigskip
\par
\begin{table}[h]
\begin{center}
\begin{tabular}{| l | l | p{60mm} |}\hline
\textbf{Campo}&\textbf{Tipo}&\textbf{Descripción} \\ \hline
id\_Email\_openSocial & INTEGER & Identificador único de la tabla. \\ \hline
ADDRESS & VARCHAR(100) & Dirección de correo electrónico. \\ \hline
TYPE & VARCHAR(45) & Descripción del tipo de cuenta de correo electrónico. \\ \hline
user\_openSocial\_id\_user\_openSocial & VARCHAR(100) & Identificador de su usuario en OpenSocial. \\ \hline
\end{tabular}
\end{center}
\caption{Tabla Email\_openSocial} \label{tabEmailOpenSocial}
\end{table}

\subsection{Tabla Gender\_openSocial}
Será una tabla maestra que definirá el sexo del usuario en OpenSocial. Sus valores serán los siguientes:
\begin{enumerate}
\item Valor \textbf{vacío}: Indicará que el usuario no ha definido esta característica en su perfil.
\item Valor \textbf{MALE}: Sexo masculino.
\item Valor \textbf{FEMALE}: Sexo femenino.
\end{enumerate}
\bigskip
\par
La tabla \ref{tabGenderOpenSocial} mostrará su organización física.
\bigskip
\par
\begin{table}[h]
\begin{center}
\begin{tabular}{| l | l | p{60mm} |}\hline
\textbf{Campo}&\textbf{Tipo}&\textbf{Descripción} \\ \hline
id\_Gender\_openSocial & INTEGER & Identificador único de la tabla. \\ \hline
DESCRIPTION & VARCHAR(45) & Descripción para cada uno de los valores anteriores. \\ \hline
\end{tabular}
\end{center}
\caption{Tabla Gender\_openSocial} \label{tabGenderOpenSocial}
\end{table}

\subsection{Tabla Presence\_openSocial}
Esta tabla indicará el estado actual del usuario en OpenSocial. Será también una tabla maestra que contendrá valores fijos que serán referenciados desde la tabla \textbf{user\_openSocial}. Sus valores serán los siguientes:
\begin{enumerate}
\item Valor \textbf{vacío}: Indicará que el usuario no ha definido esta característica en su perfil.
\item Valor \textbf{AWAY}: El usuario está conectado pero se encuentra ausente.
\item Valor \textbf{CHAT}: El usuario se encuentra manteniendo una conversación por chat.
\item Valor \textbf{DND}: Son las siglas `Do Not Disturb', por lo que el usuario estará en estado `No disponible'.
\item Valor \textbf{OFFLINE}: El usuario se encuentra desconectado actualmente.
\item Valor \textbf{ONLINE}: El usuario se encuentra conectado actualmente.
\item Valor \textbf{XA}: Son las siglas `Extended Away', por lo que el usuario se encontrará en un rango temporal elevado de inactividad aunque se encuentre conectado.
\end{enumerate}
\bigskip
\par
La tabla \ref{tabPresenceOpenSocial} mostrará su organización física.
\bigskip
\par
\begin{table}[h]
\begin{center}
\begin{tabular}{| l | l | p{60mm} |}\hline
\textbf{Campo}&\textbf{Tipo}&\textbf{Descripción} \\ \hline
id\_Presence\_openSocial & INTEGER & Identificador único de la tabla. \\ \hline
DESCRIPTION & VARCHAR(45) & Descripción para cada uno de los valores anteriores. \\ \hline
\end{tabular}
\end{center}
\caption{Tabla Presence\_openSocial} \label{tabPresenceOpenSocial}
\end{table}

\subsection{Tabla Drinker\_openSocial}
Esta tabla contendrá todos los posibles valores que existen en OpenSocial para describir el posible grado de alcoholimo del usuario. Al igual que todas las tablas de valores anteriores, será una tabla maestra referenciada desde la tabla \textbf{user\_openSocial}. Los valores definidos para ella serán los siguientes:
\begin{enumerate}
\item Valor \textbf{vacío}: Indicará que el usuario no ha definido esta característica en su perfil.
\item Valor \textbf{NO}: Indica que el usuario no bebe.
\item Valor \textbf{OCCASIONALLY}: Indica que el usuario no es bebedor pero puede beber ocasionalmente.
\item Valor \textbf{QUIT}: Indica que el usuario era bebedor pero actualmente lo ha dejado.
\item Valor \textbf{QUITTING}: Indica que el usuario está dejando de beber.
\item Valor \textbf{REGULARLY}: Indica que el usuario bebe de forma regular.
\item Valor \textbf{SOCIALLY}: Indica que el usuario bebe unicamente en reuniones sociales.
\item Valor \textbf{YES}: Indica que el usuario es bebedor.
\item Valor \textbf{HEAVILY}: Indica que es bebedor en exceso.
\end{enumerate}
\bigskip
\par
La tabla \ref{tabDrinkerOpenSocial} mostrará su organización física.
\bigskip
\par
\begin{table}[h]
\begin{center}
\begin{tabular}{| l | l | p{60mm} |}\hline
\textbf{Campo}&\textbf{Tipo}&\textbf{Descripción} \\ \hline
id\_Drinker\_openSocial & INTEGER & Identificador único de la tabla. \\ \hline
DESCRIPTION & VARCHAR(45) & Descripción para cada uno de los valores anteriores. \\ \hline
\end{tabular}
\end{center}
\caption{Tabla Drinker\_openSocial} \label{tabDrinkerOpenSocial}
\end{table}

\subsection{Tablas de direcciones Address\_openSocial y Organization\_ openSocial}
Estas tablas serán las encargadas de almacenar cualquier dirección de interés relacionada con el usuario. 
\bigskip
\par
La tabla \textbf{Address\_openSocial} almacenará toda la información respecto a la dirección física de un usuario tal como la dirección donde vive o su lugar de trabajo. Estará relacionada con la tabla \textbf{user\_openSocial} mediante una relación 1 a N y su estructura física puede verse en la tabla \ref{tabAddressOpenSocial}.
\bigskip
\par
\begin{table}[h]
\begin{center}
\begin{tabular}{| l | l | p{60mm} |}\hline
\textbf{Campo}&\textbf{Tipo}&\textbf{Descripción} \\ \hline
id\_Address\_openSocial & VARCHAR(100) & Identificador único de la tabla. \\ \hline
COUNTRY & VARCHAR(100) & País. \\ \hline
EXTENDED\_ADDRESS & VARCHAR(150) &  Dirección postal para direcciones largas.\\ \hline
LATITUDE & INTEGER & Linea de latitud del lugar. \\ \hline
LOCALITY & VARCHAR(100) & Localidad. \\ \hline
LONGITUDE & INTEGER &  Linea de longitud del lugar. \\ \hline
PO\_BOX & VARCHAR(10) &  Apartado postal. \\ \hline
POSTAL\_CODE & VARCHAR(10) &  Codigo postal. \\ \hline
REGION & VARCHAR(100) & Región del lugar. \\ \hline
STREET\_ADDRESS & VARCHAR(150) & Dirección postal.  \\ \hline
TYPE & VARCHAR(45) &  Tipo de la dirección( casa, trabajo, etc.). \\ \hline
UNSTRUCTURED & VARCHAR(200) & Si no se ha guardado la información estructuradamente, este campo almacenará toda la información de la dirección del usuario. \\ \hline
\end{tabular}
\end{center}
\caption{Tabla Address\_openSocial} \label{tabAddressOpenSocial}
\end{table}
\bigskip
\par
La tabla \textbf{Organization\_openSocial} será una extensión de la tabla anterior. Esta tabla almacenará información sobre lugares de trabajo o lugares de estudio del usuario y utilizará la tabla anterior para almacenar su dirección, mientras que la tabla Organization\_openSocial almacenará datos propios del trabajo. Estará relacionada con la tabla \textbf{user\_openSocial} de dos maneras:
\begin{enumerate}
\item Utilizará una relacion N a N para almacenar todos los trabajos del usuario.
\item Utilizará también otra relación N a N para almacenar su historial académico sobre los lugares donde estudió.
\end{enumerate}
\bigskip
\par
Su estructura física podrá verse en la tabla \ref{tabOrganizationOpenSocial}.
\bigskip
\par
\begin{table}[h]
\begin{center}
\begin{tabular}{| p{65mm} | l | p{60mm} |}\hline
\textbf{Campo}&\textbf{Tipo}&\textbf{Descripción} \\ \hline
id\_Organization\_openSocial & INTEGER & Identificador único de la tabla. \\ \hline
Address\_openSocial\_id\_Address\_ openSocial & VARCHAR(10) & Referencia a la entrada en la tabla Address\_openSocial en la que está almacenada su dirección. \\ \hline
DESCRIPTION & VARCHAR(100) & Descripción del trabajo. \\ \hline
END\_DATE & DATE &  Fecha fin del trabajo.\\ \hline
FIELD & VARCHAR(100) & Mercado de trabajo. \\ \hline
NAME & VARCHAR(100) &  Nombre de la empresa. \\ \hline
SALARY & VARCHAR(45) &  Salario. \\ \hline
START\_DATE & DATE &  Fecha de inicio del trabajo. \\ \hline
SUB\_FIELD & VARCHAR(45) & Campo del mercado de trabajo. \\ \hline
TITLE & VARCHAR(45) & Puesto de trabajo que ostenta. \\ \hline
WEB\_PAGE & VARCHAR(45) & Página web de la empresa. \\ \hline
\end{tabular}
\end{center}
\caption{Tabla Organization\_openSocial} \label{tabOrganizationOpenSocial}
\end{table}

\subsection{Tabla Friends\_openSocial}
Esta tabla contendrá todos los amigos del usuario en OpenSocial. Estará relacionada N a N con la tabla \textbf{user\_openSocial} ya que un amigo puede tener N amigos y, además, coincidir con varios amigos de otro usuario de OpenSocial con M amigos. Su estructura física se muestra en la tabla \ref{tabFriendsOpenSocial}.
\bigskip
\par
\begin{table}[h]
\begin{center}
\begin{tabular}{| l | l | p{60mm} |}\hline
\textbf{Campo}&\textbf{Tipo}&\textbf{Descripción} \\ \hline
USER\_UID & VARCHAR(45) & Identificador único del amigo en OpenSocial. \\ \hline
GROUP\_ID & VARCHAR(45) & Grupo de amigos al que pertenece. \\ \hline
NETWORK\_DISTANCE & INTEGER & Parámetro asignado por OpenSocial. Si el contacto pertenece a un grupo de amigos con los que tiene contacto frecuente, se le asignará una distancia menor, siendo mayor en caso contrario. \\ \hline
\end{tabular}
\end{center}
\caption{Tabla Friends\_openSocial} \label{tabFriendsOpenSocial}
\end{table}

\subsection{Tabla url\_openSocial}
Esta tabla contendrá todos los posibles links que sean de interés para el usuario. Todos aquellos enlaces de video o de cualquier otro tipo deberán tener una entrada en ésta tabla. Estará relacionada con la tabla \textbf{user\_openSocial} de las siguientes maneras:
\begin{enumerate}
\item Una relación N a N para guardar todos los enlaces con referencias a videos.
\item Una relación N a N para guardar todos los enlaces con refecencias a canciones de música.
\end{enumerate}
\bigskip
\par
La tabla \ref{tabUrlOpenSocial} mostrará su estructura física.
\begin{table}[h]
\begin{center}
\begin{tabular}{| l | l | p{60mm} |}\hline
\textbf{Campo}&\textbf{Tipo}&\textbf{Descripción} \\ \hline
id\_url\_openSocial & INTEGER & Identificador único de la tabla. \\ \hline
ADDRESS & VARCHAR(100) & Dirección url. \\ \hline
LINK\_TEST & VARCHAR(100) & Dirección url test en caso de error. \\ \hline
TYPE & VARCHAR(100) & Tipo de url. \\ \hline
\end{tabular}
\end{center}
\caption{Tabla url\_openSocial} \label{tabUrlOpenSocial}
\end{table}

\subsection{Tabla Name\_openSocial}
Esta tabla contendrá todo el historial de nombres u apodos que el usuario vaya usando en su perfil de OpenSocial. Estará relacionada con la tabla \textbf{user\_openSocial} con una relación N a N. Su estructura física será la mostrada en la tabla \ref{tabNameOpenSocial}.
\bigskip
\par
\begin{table}
\begin{center}
\begin{tabular}{| l | l | p{60mm} |}\hline
\textbf{Campo}&\textbf{Tipo}&\textbf{Descripción} \\ \hline
id\_Name\_openSocial & INTEGER & Identificador único de la tabla. \\ \hline
ADDITIONAL\_NAME & VARCHAR(45) & Nombre adicional del usuario. \\ \hline
FAMILY\_NAME & VARCHAR(45) & Nombre con el que se conoce familiarmente al usuario. \\ \hline
GIVEN\_NAME & VARCHAR(45) & Apodo del usuario. \\ \hline
HONORIFIC\_PREFIX & VARCHAR(45) & Prefijo honorífico del usuario. \\ \hline
HONORIFIC\_SUFFIX & VARCHAR(45) & Sufijo honorífico del usuario. \\ \hline
UNSTRUCTURED & VARCHAR(45) & Campo que almacena toda la información sin estructurar. \\ \hline
\end{tabular}
\end{center}
\caption{Tabla Name\_openSocial} \label{tabNameOpenSocial}
\end{table}

\subsection{Tabla Message\_openSocial}
Esta tabla contendrá información sobre cualquier mensaje que el usuario reciba en su perfil de OpenSocial. La tabla \textbf{Message\_Type\_openSocial} determina, mediante un conjunto de valores referenciados por la tabla \textbf{Message\_openSocial}, qué tipo de mensaje se está almacenando. Sus posibles valores serán los siguientes:
\begin{enumerate}
\item Valor \textbf{vacío}: Indicará que el tipo de mensaje no está definido.
\item Valor \textbf{EMAIL}: Indica que el mensaje es un email.
\item Valor \textbf{NOTIFICATION}: Indica que el mensaje es una notificación de alguna aplicación.
\item Valor \textbf{PRIVATE\_MESSAGE}: Indica que el mensaje es un mensaje privado.
\item Valor \textbf{PUBLIC\_MESSAGE}: Indica que el mensaje es un mensaje público, como puede ser un mensaje en el muro.
\end{enumerate}
\bigskip
\par
Su estructura física puede observarse en la tabla \ref{tabMessageTypeOpenSocial}.
\begin{table}[h]
\begin{center}
\begin{tabular}{| l | l | p{60mm} |}\hline
\textbf{Campo}&\textbf{Tipo}&\textbf{Descripción} \\ \hline
id\_Message\_Type\_openSocial & INTEGER & Identificador único de la tabla. \\ \hline
DESCRIPTION & VARCHAR(45) & Descripción para cada uno de los valores anteriores. \\ \hline
\end{tabular}
\end{center}
\caption{Tabla Message\_Type\_openSocial} \label{tabMessageTypeOpenSocial}
\end{table}
\bigskip
\par
Tras haber definido el tipo de mensaje, la tabla \textbf{Message\_openSocial} estará relacionada con la tabla \textbf{user\_openSocial} con una relación 1 a N. Su estructura física puede observarse en la tabla \ref{tabMessageOpenSocial}.
\bigskip
\par
\begin{table}[h]
\begin{center}
\begin{tabular}{| p{65mm}  | l | p{60mm} |}\hline
\textbf{Campo}&\textbf{Tipo}&\textbf{Descripción} \\ \hline
id\_Message\_openSocial & INTEGER & Identificador único de la tabla. \\ \hline
BODY & VARCHAR(1000) & Cuerpo del mensaje. \\ \hline
BODY\_ID & VARCHAR(1000) & Cuerpo del mensaje codificado mediante templates. \\ \hline
TITLE & VARCHAR(200) & Título del mensaje. \\ \hline
TITLE\_ID & VARCHAR(200) & Título del mensaje codificado mediante templates. \\ \hline
Message\_Type\_openSocial\_id\_ Message\_Type\_openSocial & INTEGER & Tipo del mensaje. \\ \hline
user\_openSocial\_id\_user\_ openSocial & VARCHAR(100) & Usuario de OpenSocial al que pertenece el mensaje. \\ \hline
\end{tabular}
\end{center}
\caption{Tabla Message\_openSocial} \label{tabMessageOpenSocial}
\end{table}

\bigskip
\par
\section{Módulo funcional de Facebook}
La máxima carga de estudio, análisis y diseño se ha desarrollado sobre la Red Social \textbf{Facebook}. Todas las decisiones que se han tomado han sido para emular, lo más exactamente posible, al entorno de procesamiento de datos de ésta Red Social. Todas las relaciones, al igual que en el diseño de OpenSocial, se centran sobre una única tabla, denominada \textbf{user\_facebook}, la cual controlará todos los datos de un usuario y harán que todos ellos sean consistentes para ser objeto de estudio posteriormente.

\subsection{Tabla user\_facebook}
Es la tabla maestra de datos de Facebook. Al igual que la tabla \textbf{user\_openSocial}, descenderá directamente de la tabla \textbf{user} y aglutinará en ella todas las relaciones con el resto de tablas que conforman todos los datos de usuario de un perfil de Facebook. Estas relaciones serán las siguientes:
\begin{enumerate}
\item Su historial de trabajo estará gestionado mediante una relación 1 a N con la tabla \textbf{work\_history\_facebook}.
\item Todo su historial academico se gestionará mediante una relación 1 a N con la tabla \textbf{education\_history\_facebook}.
\item Sus amigos y sus datos de contacto serán gestionados mediante una relación N a N con la tabla \textbf{friends\_facebook}.
\item Toda su actividad en el muro de Facebook será gestionado mediante una relación N a N con la tabla \textbf{stream\_facebook}, que a su vez contará con más relaciones que ayudarán a recrear todo el muro de Facebook con todas sus características.
\item Todas sus vinculaciones e ideas políticas serán gestionadas mediante una relación 1 a N con la tabla \textbf{affiliations\_facebook}.
\item Sus relaciones familiares serán gestionadas por una relación N a N con la tabla \textbf{family\_facebook}.
\item Todas sus direcciones postales serán gestionadas por dos relaciones N a N con la tabla \textbf{location\_facebook} que serán explicadas en el punto correspondiente.
\end{enumerate}
\bigskip
\par
Para gestionar todas estas relaciones, la estructura física necesaria será la mostrada en la tabla \ref{tabUserFacebook}.
\bigskip
\par
\begin{center}
\begin{longtable}{|l|l|p{65mm} |}

%Cabecera y primera hoja de la tabla
\caption{Tabla user\_facebook} \label{tabUserFacebook}\\
\hline \multicolumn{1}{|l|}{\textbf{Campo}} & \multicolumn{1}{l|}{\textbf{Tipo}} & \multicolumn{1}{p{65mm} |}{\textbf{Descripción}} \\ 
\hline 
\endfirsthead

%Cabecera y resto de hojas de la tabla
\hline \multicolumn{1}{|l|}{\textbf{Campo}} & \multicolumn{1}{l|}{\textbf{Tipo}} & \multicolumn{1}{p{65mm}|}{\textbf{Descripción}} \\ \hline 
\endhead

\hline
id\_user\_facebook & VARCHAR(100) & Identificador único de la tabla. \\ \hline
FIRST\_NAME & VARCHAR(100) & Nombre. \\ \hline
MIDDLE\_NAME & VARCHAR(100) & Segundo nombre( para usuarios de Estados Unidos). \\ \hline
LAST\_NAME & VARCHAR(100) & Apellidos. \\ \hline
NAME & VARCHAR(200) & Nombre completo. \\ \hline
PIC\_SMALL & VARCHAR(450) & URL de la foto de perfil a tamaño pequeño( con un ancho máximo de 50 px y un alto máximo de 150px). \\ \hline
PIC\_BIG & VARCHAR(450) & URL de la foto de perfil a tamaño grande( con un ancho máximo de 200 px y un alto máximo de 600px). \\ \hline
PIC\_SQUARE & VARCHAR(450) & URL de la foto de perfil formateada y encuadrada( con un máximo, tanto de ancho como de alto, de 50 px). \\ \hline
PIC & VARCHAR(450) & URL de la foto de perfil( con un ancho máximo de 100 px y un alto máximo de 300px). \\ \hline
PROFILE\_UPDATE\_TIME & TIME & Fecha en la que se actualizó por última vez el perfil del usuario. \\ \hline
BIRTHDAY & VARCHAR(450) & Fecha de nacimiento. \\ \hline
BIRTHDAY\_DATE & VARCHAR(450) &  Fecha de nacimiento en formato americano. \\ \hline
SIGNIFICANT\_OTHER\_ID & VARCHAR(450) &  Id de la persona con la que tiene una relación(novio, novia, marido, etc.)\\ \hline
HS1\_NAME & VARCHAR(450) &  Nombre de su primer instituto. \\ \hline
HS2\_NAME & VARCHAR(450) &  Nombre de su segundo instituto. \\ \hline
GRAD\_YEAR & INTEGER &  Año de graduación en el instituto. \\ \hline
HS1\_ID & VARCHAR(450) &  Id de su primer instituto. \\ \hline
HS2\_ID & VARCHAR(450) &  Id de su segundo instituto. \\ \hline
NOTES\_COUNT & INTEGER & Numero de notas creadas por el usuario. \\ \hline
WALL\_COUNT & INTEGER &  Numero de mensajes de muro escritos por el usuario. \\ \hline
ONLINE\_PRESENCE & VARCHAR(450) &  Información sobre el estado de chat del usuario. \\ \hline
LOCALE & VARCHAR(450) &  Codigo que indica el idioma configurado por el usuario. \\ \hline
PROXIED\_EMAIL & VARCHAR(450) &  Email interno cifrado del usuario que Facebook utiliza internamente. \\ \hline
PROFILE\_URL & VARCHAR(450) &  URL del perfil del usuario. \\ \hline
EMAIL\_HASHES & VARCHAR(450) &  Contiene los emails confirmados del usuario. \\ \hline
PIC\_SMALL\_WITH\_LOGO & VARCHAR(450) & URL de la foto de perfil a tamaño pequeño( con un ancho máximo de 50 px y un alto máximo de 150px) con el logo de Facebook.  \\ \hline
PIC\_BIG\_WITH\_LOGO & VARCHAR(450) &  URL de la foto de perfil a tamaño grande( con un ancho máximo de 200 px y un alto máximo de 600px) con el logo de Facebook.\\ \hline
PIC\_SQUARE\_WITH\_LOGO & VARCHAR(450) &  URL de la foto de perfil formateada y encuadrada( con un máximo, tanto de ancho como de alto, de 50 px) con el logo de Facebook. \\ \hline
PIC\_WITH\_LOGO & VARCHAR(450) & URL de la foto de perfil( con un ancho máximo de 100 px y un alto máximo de 300px) con el logo de Facebook. \\ \hline
ALLOWED\_RESTRICTIONS & VARCHAR(450) & Lista, separada por ;, de restricciones demográficas. Actualmente, sólo devuelve el valor \textbf{alcohol} \\ \hline
VERIFIED & TINYINT(1) & Indica si el perfil de Facebook ha sido verificado por su usuario. \\ \hline
PROFILE\_BLURB & VARCHAR(450) & Contiene el comentario del usuario acerca de su perfil en Facebook. \\ \hline
USERNAME & VARCHAR(450) & Nombre de usuario dentro de Facebook. \\ \hline
WEBSITE & VARCHAR(450) & URL al sitio web propio y externo del usuario. \\ \hline
IS\_BLOCKED & TINYINT(1) & Indica si el perfil de Facebook está bloqueado. \\ \hline
CONTACT\_EMAIL & VARCHAR(450) & Email de contacto.  \\ \hline
EMAIL & VARCHAR(450) & Email del usuario. \\ \hline
MEETING\_FOR & VARCHAR(45) & Lista de preferencias a la hora de conocer a alguien. \\ \hline
MEETING\_SEX & VARCHAR(45) & Lista de preferencias a la hora de conocer a alguien para una relación. \\ \hline
\end{longtable}
\end{center}

\subsection{Tabla work\_history\_facebook}
Será la tabla que almacenará toda la vida laboral que el usuario introduce en su perfil de Facebook. Estará gestionada por una relación 1 a N con la tabla \textbf{user\_facebook}. Su estructura física está representada en la tabla \ref{tabWorkHistoryFacebook}.
\bigskip
\par
\begin{table}[h]
\begin{center}
\begin{tabular}{| l | l | p{60mm} |}\hline
\textbf{Campo}&\textbf{Tipo}&\textbf{Descripción} \\ \hline
id\_work\_history\_facebook & INTEGER & Identificador único de la tabla. \\ \hline
LOCATION & VARCHAR(100) & Lugar de trabajo. \\ \hline
COMPANY\_NAME & VARCHAR(100) &  Nombre de la empresa de trabajo.\\ \hline
DESCRIPTION & VARCHAR(100) & Descripción del puesto de trabajo. \\ \hline
POSITION & VARCHAR(100) &  Cargo en el puesto de trabajo. \\ \hline
START\_DATE & VARCHAR(7) &  Fecha de inicio del trabajo. \\ \hline
END\_DATE & VARCHAR(7) & Fecha fin del trabajo. \\ \hline
user\_facebook\_id\_user\_facebook & VARCHAR(100) & Usuario de Facebook con el que está relacionado. \\ \hline
\end{tabular}
\end{center}
\caption{Tabla work\_history\_facebook} \label{tabWorkHistoryFacebook}
\end{table}

\subsection{Tabla education\_history\_facebook}
Esta tabla contendrá los datos académicos universitarios del usuario de Facebook. Estará gestionada por una relación 1 a N con la tabla \textbf{user\_facebook}. Su estructura física estará representada en la tabla \ref{tabEducationHistoryFacebook}.
\bigskip
\par
\begin{table}[h]
\begin{center}
\begin{tabular}{| l | l | p{60mm} |}\hline
\textbf{Campo}&\textbf{Tipo}&\textbf{Descripción} \\ \hline
id\_education\_history\_facebook & INTEGER & Identificador único de la tabla. \\ \hline
YEAR & VARCHAR(5) & Año de inicio. \\ \hline
NAME & VARCHAR(100) &  Nombre del centro educativo.\\ \hline
DEGREE & VARCHAR(100) & Nombre del estudio universitario. \\ \hline
CONCENTRATIONS\_0 & VARCHAR(200) & Campo para comentarios y experiencias del primer año. \\ \hline
CONCENTRATIONS\_1 & VARCHAR(200) & Campo para comentarios y experiencias del segundo año. \\ \hline
CONCENTRATIONS\_2 & VARCHAR(200) & Campo para comentarios y experiencias del tercer año. \\ \hline
user\_facebook\_id\_user\_facebook & VARCHAR(100) & Usuario de Facebook con el que está relacionado. \\ \hline
\end{tabular}
\end{center}
\caption{Tabla education\_history\_facebook} \label{tabEducationHistoryFacebook}
\end{table}

\subsection{Tabla friends\_facebook}
Será la tabla donde se almacenarán todas las amistades de un determinado perfil. Estará gestionada con una relación N a N con la tabla \textbf{user\_facebook}. Se almacenan datos básicos como la edad de un amigo para, posteriormente, ser analizado por el filtro de control de amistades. Su estructura física estará representada en la tabla \ref{tabFriendsFacebook}.
\bigskip
\par
\begin{table}[h]
\begin{center}
\begin{tabular}{| l | l | p{60mm} |}\hline
\textbf{Campo}&\textbf{Tipo}&\textbf{Descripción} \\ \hline
USER\_UID & VARCHAR(45) & Identificador único de Facebook del amigo del usuario. \\ \hline
USER\_NAME & VARCHAR(45) & Nombre del amigo. \\ \hline
USER\_BIRTHDAY & VARCHAR(45) &  Fecha de nacimiento del amigo.\\ \hline
USER\_PIC & VARCHAR(200) &  URL a la foto de perfil del amigo en Facebook.\\ \hline
\end{tabular}
\end{center}
\caption{Tabla friends\_facebook} \label{tabFriendsFacebook}
\end{table}

\subsection{Tabla stream\_facebook}
En cuanto a diseño, la tabla \textbf{stream\_facebook} es la más compleja de construir, no sólo estructuralmente, sino funcionalmente, ya que va a constituir uno de los pilares básicos de análisis para los filtros configurados actualmente y los que se diseñen en un futuro. El análisis de un comentario del muro de un usuario conlleva una serie de relaciones que analizamos a continuación:
\begin{enumerate}
\item Llevará una relación N a N con la tabla \textbf{user\_facebook} para tener controlados todas las entradas en el muro del usuario en todo momento.
\item Estará relacionado 1 a N con la tabla \textbf{likes\_facebook}, la cual contabilizará todos los comentarios positivos de los amigos que ven cada comentario.
\item El grado de privacidad del comentario estará gestionado por una relación 1 a N con la tabla \textbf{privacy\_facebook}.
\item Los comentarios a la entrada del muro estarán almacenados mediante una relación 1 a N con la tabla \textbf{comments\_facebook}.
\item Todos los links que se adjunten a la entrada del muro estarán almacenados mediante la relación 1 a N con la tabla \textbf{action\_links\_facebook}.
\end{enumerate}
\bigskip
\par
Por todo ello, la estructura física de la tabla \textbf{stream\_facebook} seguirá la estructura de la tabla \ref{tabStreamFacebook}.
\bigskip
\par
\begin{center}
\begin{longtable}{|l|l|p{65mm} |}

%Cabecera y primera hoja de la tabla
\caption{Tabla stream\_facebook} \label{tabStreamFacebook}\\
\hline \multicolumn{1}{|l|}{\textbf{Campo}} & \multicolumn{1}{l|}{\textbf{Tipo}} & \multicolumn{1}{p{65mm} |}{\textbf{Descripción}} \\ 
\hline 
\endfirsthead

%Cabecera y resto de hojas de la tabla
\hline \multicolumn{1}{|l|}{\textbf{Campo}} & \multicolumn{1}{l|}{\textbf{Tipo}} & \multicolumn{1}{p{65mm}|}{\textbf{Descripción}} \\ \hline 
\endhead

\hline
post\_id & VARCHAR(100) & Identificador único del post. \\ \hline
viewer\_id & VARCHAR(100) & Usuario al que va dirigido el comentario. \\ \hline
app\_id & VARCHAR(100) & En caso de que el comentario sea escrito por una aplicación, este campo almacenará el ID de Facebook de dicha aplicación. \\ \hline
source\_id & VARCHAR(100) & Usuario propietario del post. \\ \hline
updated\_time & TIMESTAMP & Fecha de actualización del post. Se actualizará cada vez que a un usuario le guste el post o se añadan nuevos comentarios a él. \\ \hline
created\_time & TIMESTAMP & Fecha de creación del post. \\ \hline
filter\_key & VARCHAR(100) & Metadatos del post. \\ \hline
attribution & VARCHAR(100) & Nombre completo de la aplicación que genera el post. \\ \hline
actor\_id & VARCHAR(100) & ID del usuario que genera el post. \\ \hline
target\_id & VARCHAR(100) & Usuario al que referencia el post. \\ \hline
message & MEDIUMTEXT & Mensaje del post. \\ \hline
app\_data & VARCHAR(10000) &  Datos de la aplicación que genera el post. \\ \hline
attachment & VARCHAR(10000) &  Información del archivo generado por el post en Facebook. \\ \hline
type & VARCHAR(100) &  Tipo del post dependiendo de quien lo cree o lo actualice. \\ \hline
permalink & VARCHAR(300) &  URL asociada al post. \\ \hline
xid & VARCHAR(45) &  ID asociado a Live Stream Box. \\ \hline
\end{longtable}
\end{center}

\subsubsection{Tabla likes\_facebook}
Esta tabla almacenará información acerca del numero de \textbf{likes} que recibe un post del muro de un usuario. Tendrá dos tipos de relaciones:
\begin{enumerate}
\item Una relación 1 a N con la tabla \textbf{stream\_facebook} que controlará la información acerca de ese comentario en concreto.
\item Una relación 1 a N con la tabla \textbf{friends\_likes\_facebook} que controlará los datos de cada amigo al que le ha gustado la entrada de post.
\end{enumerate}
\bigskip
\par
Con todas estas referencias, la estructura física que conforma la tabla \textbf{likes\_facebook} estará representada en la tabla \ref{tabLikesFacebook}.
\bigskip
\par
\begin{table}[h]
\begin{center}
\begin{tabular}{| l | l | p{60mm} |}\hline
\textbf{Campo}&\textbf{Tipo}&\textbf{Descripción} \\ \hline
id\_likes\_facebook & INTEGER & Identificador único de la tabla. \\ \hline
HREF & VARCHAR(100) & URL del post. \\ \hline
COUNT & INTEGER &  Numero total de likes.\\ \hline
USER\_LIKES & TINYINT(1) & Indica si a alguien le ha gustado el post. \\ \hline
CAN\_LIKE & TINYINT(1) & Indica si a alguien puede gustarle el post. \\ \hline
stream\_facebook\_post\_id & VARCHAR(100) & Post de la tabla stream\_facebook con el que está relacionado. \\ \hline
\end{tabular}
\end{center}
\caption{Tabla likes\_facebook} \label{tabLikesFacebook}
\end{table}

\bigskip
\par
Para almacenar cada uno de los usuarios a los que les ha gustado el post, se ha diseñado la tabla \textbf{friends\_likes\_facebook}, cuya estructura física está representada en la tabla \ref{tabFriendsLikesFacebook}.
\bigskip
\par
\begin{table}[h]
\begin{center}
\begin{tabular}{| l | l | p{60mm} |}\hline
\textbf{Campo}&\textbf{Tipo}&\textbf{Descripción} \\ \hline
id\_friends\_likes\_facebook & INTEGER & Identificador único de la tabla. \\ \hline
UID & VARCHAR(45) & UID del amigo de Facebook al que le ha gustado el post. \\ \hline
likes\_facebook\_id\_likes\_facebook & INTEGER & Entrada de la tabla likes\_facebook con la que está relacionado. \\ \hline
\end{tabular}
\end{center}
\caption{Tabla friends\_likes\_facebook} \label{tabFriendsLikesFacebook}
\end{table}

\subsubsection{Tabla privacy\_facebook}
Esta tabla controla los usuarios que pueden visualizar el post del usuario de Facebook. Estará gestionado por una relación 1 a N con la tabla \textbf{stream\_facebook}. Su estructura física puede observarse en la tabla \ref{tabPrivacyFacebook}.
\bigskip
\par
\begin{table}[h]
\begin{center}
\begin{tabular}{| l | l | p{60mm} |}\hline
\textbf{Campo}&\textbf{Tipo}&\textbf{Descripción} \\ \hline
id\_privacy\_facebook & INTEGER & Identificador único de la tabla. \\ \hline
UID & VARCHAR(45) & UID del amigo de Facebook que puede visualizar el post. \\ \hline
stream\_facebook\_post\_id & VARCHAR(100) & Post de la tabla stream\_facebook con el que está relacionado. \\ \hline
\end{tabular}
\end{center}
\caption{Tabla privacy\_facebook} \label{tabPrivacyFacebook}
\end{table}

\subsubsection{Tabla comments\_facebook}
Esta tabla almacenará todas las respuestas a los post que se hayan generado en el muro del usuario. Tendrá las siguientes relaciones:
\begin{enumerate}
\item Seguirá una relación 1 a N con la tabla \textbf{stream\_facebook} ya que un post podrá tener N respuestas.
\item Se relacionará 1 a N con la tabla \textbf{comment\_facebook}, la cual almacenará todos los datos de cada una de las respuestas, cuando fue creado y cuando fue modificado.
\end{enumerate}
\bigskip
\par
Con estos datos, su estructura física será la representada en la tabla \ref{tabCommentsFacebook}.

\bigskip
\par
\begin{table}[h]
\begin{center}
\begin{tabular}{| l | l | p{60mm} |}\hline
\textbf{Campo}&\textbf{Tipo}&\textbf{Descripción} \\ \hline
id\_comments\_facebook & INTEGER & Identificador único de la tabla. \\ \hline
CAN\_REMOVE & TINYINT(1) & Indica si el post puede ser borrado. \\ \hline
CAN\_POST & TINYINT(1) & Indica si el post puede ser comentado. \\ \hline
stream\_facebook\_post\_id & VARCHAR(100) & Post de la tabla stream\_facebook con el que está relacionado. \\ \hline
\end{tabular}
\end{center}
\caption{Tabla comments\_facebook} \label{tabCommentsFacebook}
\end{table}

Como se ha indicado anteriormente, la tabla \textbf{comments\_facebook} se relacionará 1 a N con la tabla \textbf{comment\_facebook}, la cual almacenará información de cada uno de los comentarios a los post y seguirá la estructura física de la tabla \ref{tabCommentFacebook}.
\bigskip
\par
\begin{table}[h]
\begin{center}
\begin{tabular}{| p{50mm}  | l | p{60mm} |}\hline
\textbf{Campo}&\textbf{Tipo}&\textbf{Descripción} \\ \hline
ID & VARCHAR(100) & Identificador único de la tabla. \\ \hline
XID & VARCHAR(100) & ID asociado a Live Stream Box. \\ \hline
POST\_ID & VARCHAR(100) & ID asociado al post de la tabla stream\_facebook. \\ \hline
FROMID & VARCHAR(100) & Indica el usuario que ha generado el comentario. \\ \hline
TIME & TIMESTAMP & Fecha de creación. \\ \hline
TEXT & MEDIUMTEXT & Texto del comentario. \\ \hline
USERNAME & VARCHAR(100) & Nombre del usuario que ha generado el comentario. \\ \hline
REPLY\_XID & VARCHAR(100) & Contestación generada desde Live Stream Box. \\ \hline
comments\_facebook\_id\_ comments\_facebook & INTEGER & Post de la tabla comments\_facebook con el que está relacionado. \\ \hline
\end{tabular}
\end{center}
\caption{Tabla comment\_facebook} \label{tabCommentFacebook}
\end{table}

\subsubsection{Tabla action\_links\_facebook}
La tabla \textbf{action\_links\_facebook} mostrará información de todos aquellos post que se hayan generado con cualquier tipo de enlace a videos o clips de audio dentro del texto. Estará relacionada con la tabla \textbf{stream\_facebook} mediante una relación 1 a N. Su estructura física será la mostrada en la tabla \ref{tabActionLinksFacebook}.
\bigskip
\par
\begin{table}[h]
\begin{center}
\begin{tabular}{| l | l | p{60mm} |}\hline
\textbf{Campo}&\textbf{Tipo}&\textbf{Descripción} \\ \hline
id\_action\_links\_facebook & INTEGER & Identificador único de la tabla. \\ \hline
TEXT & VARCHAR(200) & Texto del post. \\ \hline
URL & VARCHAR(100) & URL al contenido del video o del clip de audio. \\ \hline
stream\_facebook\_post\_id & VARCHAR(100) & Post de la tabla stream\_facebook con el que está relacionado. \\ \hline
\end{tabular}
\end{center}
\caption{Tabla action\_links\_facebook} \label{tabActionLinksFacebook}
\end{table}

\subsection{Tabla affiliations\_facebook}
Esta tabla almacenará cualquier tipo de relación del perfil de un usuario de Facebook, es decir, almacenará datos de la persona con la que tenga una relación de novios, maridos, etc. Se verá afectada por las siguientes relaciones:
\begin{enumerate}
\item Seguirá una relación 1 a N con la tabla \textbf{user\_facebook}.
\item Se relacionará N a 1 con la tabla \textbf{type\_affiliations\_facebook}, la cual almacenará todos los tipos de relación que se hayan dado de alta en Facebook.
\end{enumerate}
\bigskip
\par
Con todos estos datos, su estructura física quedará de la forma indicada por la tabla \ref{tabAffiliationsFacebook}.
\bigskip
\par
\begin{table}[h]
\begin{center}
\begin{tabular}{| p{55mm}  | l | p{60mm} |}\hline
\textbf{Campo}&\textbf{Tipo}&\textbf{Descripción} \\ \hline
id\_affiliations\_facebook & INTEGER & Identificador único de la tabla. \\ \hline
YEAR & VARCHAR(10) & Año en el que comenzó la relación. \\ \hline
NAME & VARCHAR(200) & Nombre del usuario con el que está relacionado. \\ \hline
NID & INTEGER & ID del usuario de Facebook con el que está relacionado. \\ \hline
STATUS & VARCHAR(100) & Estado de la relación. \\ \hline
type\_affiliations\_facebook\_id \_type\_affiliations\_facebook & INTEGER & Tipo de relación de la tabla type\_affiliations\_facebook. \\ \hline
user\_facebook\_id\_user\_ facebook & VARCHAR(100) & Usuario de la tabla user\_facebook con el que está relacionado. \\ \hline
\end{tabular}
\end{center}
\caption{Tabla affiliations\_facebook} \label{tabAffiliationsFacebook}
\end{table}
\bigskip
\par
Para obtener el tipo de la relación, se diseñó la tabla \textbf{type\_affiliations\_facebook}, la cual indicará el tipo de relación entre ambas personas. Su estructura física se mostrará en la tabla \ref{tabTypeAffiliationsFacebook}.
\bigskip
\par
\begin{table}[h]
\begin{center}
\begin{tabular}{| l | l | p{60mm} |}\hline
\textbf{Campo}&\textbf{Tipo}&\textbf{Descripción} \\ \hline
id\_type\_affiliations\_facebook & INTEGER & Identificador único de la tabla. \\ \hline
DESCRIPTION & VARCHAR(100) & Descripción para cada uno de los tipos de relación en Facebook. \\ \hline
\end{tabular}
\end{center}
\caption{Tabla type\_affiliations\_facebook} \label{tabTypeAffiliationsFacebook}
\end{table}


\subsection{Tabla family\_facebook}
Esta tabla almacenará todas las relaciones familiares del usuario. Seguirá las siguientes relaciones:
\begin{enumerate}
\item Seguirá una relación N a N con la tabla \textbf{user\_facebook}, ya que un usuario podrá tener N relaciones familiares.
\item Se relacionará N a 1 con la tabla \textbf{relationship\_facebook}, la cual almacenará todos los tipos de relación familiar que se hayan dado de alta en Facebook.
\end{enumerate}
\bigskip
\par
Su estructura física puede observarse en la tabla \ref{tabFamilyFacebook}.
\bigskip
\par
\begin{table}[h]
\begin{center}
\begin{tabular}{| p{55mm}  | l | p{60mm} |}\hline
\textbf{Campo}&\textbf{Tipo}&\textbf{Descripción} \\ \hline
id\_family\_facebook & INTEGER & Identificador único de la tabla. \\ \hline
relationship\_facebook\_id \_relationship\_facebook & INTEGER & Tipo de relación familiar de la tabla relationship\_facebook. \\ \hline
UID & VARCHAR(45) & UID del usuario de Facebook con el que se mantiene una relación familiar. \\ \hline
NAME & VARCHAR(100) & Nombre del usuario con el que está relacionado. \\ \hline
BIRTHDAY & VARCHAR(45) & Fecha de nacimiento del usuario de Facebook con el que está relacionado. \\ \hline
\end{tabular}
\end{center}
\caption{Tabla family\_facebook} \label{tabFamilyFacebook}
\end{table}
\bigskip
\par
La tabla \textbf{relationship\_facebook} mostrará los tipos de relaciones dados de alta por el usuario en Facebook. Su estructura física será la representada en la tabla \ref{tabRelationshipFacebook}.
\bigskip
\par
\begin{table}[h]
\begin{center}
\begin{tabular}{| l | l | p{60mm} |}\hline
\textbf{Campo}&\textbf{Tipo}&\textbf{Descripción} \\ \hline
id\_relationship\_facebook & INTEGER & Identificador único de la tabla. \\ \hline
DESCRIPTION & VARCHAR(100) & Descripción para cada uno de los tipos de relación familiar en Facebook. \\ \hline
\end{tabular}
\end{center}
\caption{Tabla relationship\_facebook} \label{tabRelationshipFacebook}
\end{table}

\subsection{Tabla location\_facebook}
Esta tabla mostrará el lugar donde actualmente reside el usuario. Estará relacionado N a N con la tabla \textbf{user\_facebook} y su estructura física seguirá la estructura de la tabla \ref{tabLocationFacebook}.
\bigskip
\par
\begin{table}[H]
\begin{center}
\begin{tabular}{| l | l | p{60mm} |}\hline
\textbf{Campo}&\textbf{Tipo}&\textbf{Descripción} \\ \hline
id\_location\_facebook & INTEGER & Identificador único de la tabla. \\ \hline
CITY & VARCHAR(45) & Ciudad. \\ \hline
STATE & VARCHAR(45) & Estado o Comunidad Autónoma. \\ \hline
COUNTRY & VARCHAR(45) & País. \\ \hline
\end{tabular}
\end{center}
\caption{Tabla location\_facebook} \label{tabLocationFacebook}
\end{table}