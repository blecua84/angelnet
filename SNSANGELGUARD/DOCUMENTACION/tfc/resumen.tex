\chapter{Resumen}
SNSAngelGuard es un software desarrollado por la Universidad de Alcalá cuyo objetivo es controlar la actividad de una determinada persona o perfil dentro de la Red Social Facebook.  Una figura denominada Tutor controlará, por medio de informes generados desde SNSAngelGuard, la actividad que un usuario pueda generar en dicha red social. Los filtros actuales dentro de la aplicación permiten controlar los mensajes que se escriben dentro del muro, la edad de amigos potencialmente peligrosos, su configuración de seguridad y el número de visitas que recibe una determinada persona o perfil de Facebook. Con ésta implementación se consigue controlar determinados acosos por terceras personas que el usuario pueda recibir y a su vez, notificarse automáticamente a su Tutor. Esta aplicación se ejecutará en el entorno de Facebook y utilizará una base de datos externa que, mediante servicios RestFul, controlará el tráfico de datos del usuario y servirá como base de análisis a la aplicación.

\chapter{Abstract}
SNSAngelGuard is a softwer developed for the University of Alcala de Henares and its objetive is controling the activity of a determinate Facebook usser or his/her profile within this social network. A feature called Tutor, will control, through generated researchs from SNSAngelGuard, the activity that a usser could develop in this social network. The current filters within the aplication allow the control of the messages posted on the wall, the age of the potential dangerous friends, the security settings and the number of visits this usser or Facebook profile can recive. With this implementation the application takes control of possible harrasments by third people and at the same time, it communicates them automatically to the Tutor. This application will be run in the Facebook enviroment and it will  use an external database which, by using RestFul services, will protect the usser data traffic and will work as a base for the application analysis.
